\documentclass{article}

\usepackage{graphicx}
\usepackage{tcolorbox}
\usepackage{amsmath}
\tcbuselibrary{theorems}
\sloppy
\newtcbtheorem[number within=section]{mytheo}{Question}%
{colback=blue!5,colframe=blue!35!black,fonttitle=\bfseries}{th}

\title{Physics 212}
\date{\vspace{-5ex}}
\author{HW Problem}
\begin{document}
\maketitle

\begin{mytheo}{Electric Potential}{theoexample}
	An infinite sheet of charge is located in the $y-z$ plane at $x = 0$ and has uniform charge density \sigma_{1} = $0.59 \mu \frac{C}{m_2}$. Another infinite sheet of charge with uniform charge density $\sigma_2 = -0.35 \mu \frac{C}{m_2}$ is located at $x = c = 30$ cm. An uncharged infinite conducting slab is placed halfway in between these sheets ( i.e., between $x = 13$ cm and $x = 17$ cm).\\

1)
What is Ex(P), the x-component of the electric field at point P, located at (x,y) = (6.5 cm, 0)? \\

2)
What is $\sigma_{a}$, the charge density on the surface of the conducting slab at x = 13 cm? \\

3) What is V(R) - V(P), the potentital difference between point P and point R, located at (x,y) = (6.5 cm, -17 cm)?\\

4) What is V(S) - V(P), the potential difference between point P and point S, located at (x,y) = (23.5 cm, -17 cm)?
\end{mytheo}
\includegraphics{sadge.png}

\section{What is Ex(P), the x-component of the electric field at point P, located at (x,y) = (6.5 cm, 0)?}

\end{document}
