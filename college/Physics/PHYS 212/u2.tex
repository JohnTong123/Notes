\documentclass{article}

\usepackage{graphicx}
\usepackage{tcolorbox}
\usepackage{amsmath}
\tcbuselibrary{theorems}
\sloppy
\newtcbtheorem[number within=section]{mytheo}{Question}%
{colback=blue!5,colframe=blue!35!black,fonttitle=\bfseries}{th}

\title{Physics 212}
\date{\vspace{-5ex}}
\author{Exam 2 Study Guide}
\begin{document}
\maketitle
\section{Capacitors}
\subsection{Main Concepts}
\begin{itemize}
	\item Capacitors in series have the same current. 
	\item The formula for total capacitance in series is $\frac{1}{C_{12}} =\frac{1}{C_1} + \frac{1}{C_2}$  
	i\item Capacitors in parallel have the same voltage.
\item The formula for total capacitance in parallel is $C_{12} = C_1 + C_2$

\end{itemize}
\section{Electric Current}
\subsection{Main Concepts}
\begin{itemize}
		\item The resistance of a resistor is calculated by $R=\frac{L}{\sigma A}$
		\item R = Resistance, L = length, A = Cross-sectional Area, $\sigma =$ Resistivity.
		\item Resistors in series have the same current.
		\item The formula for total resistance in series is $R_{12} = R_1 + R_2$
		\item Resistors in parallel have the same voltage/
		\item The formula for total resistance in parallel is $\frac{1}{R_{12}} = \frac{1}{R_1} + \frac{1}{R_2}$
		\item Ohm's Law is V=IR
		\item $J = \sigma E$
		\item J is current density at a location, $\sigma$ is conductivity, E is electric field at that location
		\item Electric Current is $ I = \frac{dq}{dt}$
		\item $ P = IV$
		\item P = Power, I = Current, V = Voltage
\end{itemize}
\section{Kirchoff's Rules}
\subsection{Main Concepts}
\begin{itemize}
	\item Kirchoff's Voltage Law, In a closed loop $\Sigma\Delta V_n = 0 $
	\item Kirchoff's Current Law, at any node $\Sigma I_{in} = \Sigma I_{out}$ 
\end{itemize}
\section{RC Circuits}
\subsection{Main Concepts}
\begin{itemize}
	\item Resistors will cause capacitance to slowly gain voltage and slowly lose voltage once unplugged.
	\item Discharging Charge: $Q_0e^{-\frac{t}{RC}}$
	\item Charging Charge: $Q_{/infty}1-e^{-\frac{t}{RC}}$
	\item Discharging Current: $I_0e^{-\frac{t}{RC}}$
	\item Charging Current: $I_0e^{\frac{-t}{RC}}$
	\item Discharging Voltage: $V_0e^{-\frac{t}{RC}}$
	\item Charging Voltage: $V_{\infty}(1-e^{-\frac{t}{RC}})$


\end{itemize}
\subsection{Proof of Charge relationship}
\begin{align*}
	\Sigma V &= 0\\
	V_{r} + V_{c} &= 0\\
	\frac{dQ}{dt}R  &= - \frac{Q}{C}\\
	\frac{1}{Q}RdQ R &= -\frac{1}{C}dt\\
	\int^{final}_{0} \frac{R}{Q}dQ &= -\int \frac{1}{c}dt\\
	R\ln \frac{Q_{final}}{Q_0} &= -\frac{t}{c}\\
	\ln \frac{Q_{final}}{Q_0} &= -\frac{t}{RC}\\
	\frac{Q_{final}}{Q_0} &= e^{-\frac{t}{RC}}\\
	Q_{final} &= Q_0 e^{-\frac{t}{RC}}
\end{align*}
\section{Magnetism}
\subsection{Main Concepts}
\begin{itemize}
	\item Magnetic fields create electric Currents
	\item Magnetic Fields exert forces on electric currents (the charges in motion)
	\item $\vec F = q \vec v x \vec B$
	\item Force on a particle  = the charge of particle multiplied by the cross product of its velocity and the magnetic field
	\item Right hand rule! Index finger is v, other is B, thumb is F!
\end{itemize}

\section{Forces and Torques on Currents}
\subsection{Main Concepts}
\begin{itemize}
	\item $\vec F = I \vec L x \vec B$
	\item F = Force, I = current, L = length, B = Magnetic field
	\item $\vec \mu = NI \vec A$
	\item $\mu$ = Dipole moment N = num of coils, I = current A = Area 
	\item $\vec\tau = \vec\mu x \vec B $
	\item $\tau$ = Torque, $\mu$ = Magnetic Dipole Moment, B = Magnetic Field
	\item $W = \int\tau  * d\theta$
	\item $\Delta U = -\vec\mu\cdot\vec B$

	i 
\end{itemize}
\section{Biot-Savart Law}
\subsection{Main Concepts}
\begin{itemize}
	\item $\vec dB = \frac{\mu_0 I}{4 \pi} \frac{\vec ds x \vec r}{r^2}$
	\item \vec Infinite Line : $B= \frac{\mu_0 I}{2 \pi r}$
	\item Current Loop: $ B = \frac{\mu_0 I }{2} \frac{R^2}{(R^2 + z^2)^\frac{3}{2}}$
	\item 
\end{itemize}
\section{Ampere's Law}
\subsection{Main Concepts}
\begin{itemize}
	\item $\int \vec B \cdot \vec dl = \mu_0 I_{enclosed}$
	\item For a solenoid: $B = \frac{\mu_0nI}{2}$
\end{itemize}
\section{Motional EMF}
\subsection{Main Concepts}
\begin{itemize}
	\item $\vec F = q\vec v = \vec B$
\end{itemize}

\section{Faraday's Law}
\subsection{Main Concepts}
\begin{itemize}
	\item $\Phi = \int \vec B \cdot \vec dA$
	\item $\varepsilon = -\frac{d\Phi}{dt} = \int \vec E \cdot \vec dl$
\end{itemize}



\end{document}
