\documentclass{article}

\usepackage{graphicx}
\usepackage{tcolorbox}
\usepackage{amsmath}
\tcbuselibrary{theorems}
\sloppy
\newtcbtheorem[number within=section]{mytheo}{Question}%
{colback=blue!5,colframe=blue!35!black,fonttitle=\bfseries}{th}

\title{Physics 212}
\date{\vspace{-5ex}}
\author{HW Problem}
\begin{document}
\maketitle

\begin{mytheo}{Potential of Concentric Cylindrical Insulator and Conducting Shell}{theoexample}
	An infinitely long solid insulating cylinder of radius $a = 4.1$ cm is positioned with its symmetry axis along the z-axis as shown. The cylinder is uniformly charged with a charge density $\rho = 37\frac{\mu C}{m^3}$. Concentric with the cylinder is a cylindrical conducting shell of inner radius b = 16.6 cm, and outer radius c = 19.6 cm. The conducting shell has a linear charge density $\lambda = -0.47 \frac{\mu C}{m}.$\\

1)
What is $ E_{y}(R)$, the y-component of the electric field at point R, located a distance d = 46 cm from the origin along the y-axis as shown? \\


2) What is V(P) – V(R), the potential difference between points P and R? Point P is located at (x,y) = (46 cm, 46 cm).\\

3)What is V(c) - V(a), the potential difference between the outer surface of the conductor and the outer surface of the insulator?\\
\end{mytheo}
\includegraphics{circle}

\section{Electric Field Component}
We're gonna operate under the assumption that according to Gauss's Law that the E of an infinite line is equivalent to $2k\frac{\lambda}{r}$ or $\frac{\lambda}{2\epsilon_{0}\pi}$. \\
Since our point R is outside of both the shell and cylinder, when we're calculating electric field at R, we can just consider them as one with a collective charge of $\lambda_{total} = \lambda_{shell} + \lambda_{cylinder}$\\
It's important to note that we are solving for LINEAR CHARGE DENSITY, meaning that our calculations will result in the units of $\frac{C}{m^3}$ or with units of microcoulombs. Checking units is one of the strongest concepts in physics.  \\
Fortunately, our linear charge density for the shell is given, but we're given charge density of the volume for the cylinder. To solve for this, we multiply by the area of a cross section of the cylinder, which results in the circle with radius a. 

\begin{align*}
	\lambda_{cylinder} &= \rho * \pi a^2\\
	\lambda_{cylinder} &= 37 \frac{\mu C}{m^3} * \pi * 0.041^2 m^2\\
\lambda_{cylinder} &=  0.195\frac{\mu C}{m}
\end{align*}
Now, we go back to our sum of linear charges and solve for the $\lambda_{total}$
\begin{align*}
	\lambda_{total} &= \lambda_{shell} + \lambda_{cylinder}\\
	\lambda_{total} &= -0.47 \frac{\mu C}{m} + 0.195 \frac{\mu C}{m}\\
	\lambda_{total} &= -0.275 \frac{\mu C}{m}
\end{align*}
Now we cook. We use the gauss's law formula from the beginning to find our electric field at distance R. Also, although we're supposed to find the y component, it's essentially just the magnitude in this case, because the point lies directly above the center of the cylinder and shell.

\begin{align*}
	E_{y} &= \frac{2k\lambda}{r}\\
	E_{y} &= \frac{2 * 9 * 10^9 \frac{Nm^2}{C^2} * -0.275 * \frac{\mu C}{m} * 10^{-6} \frac{C}{\mu C}}{0.46m}\\
	E_{y} &= -10760\frac{N}{C}
\end{align*}

\section{Potential Difference between P and R}
Ok, im ngl i hate these questions but lets do it.\\
Our main way to solve for voltage is gonna be the formula\\

\[
	V = -\int^a_{b} E dl 
\]
So let's break this statement down. Essentially, Voltage is equivalent to the negative of the integral of the Electric field, from point b to a.\\
Thankfully, with Gauss's law, we know the electric field equation for an infinite line $E = 2k\frac{\lambda}{r}$. Here r, the distance from the cylinder, is equal to l. After plugging into the equation we get:\\
\[
	V = -\int^{a}_{b} 2k\frac{\lambda}{l} dl
\]
In this scenario, all of the points we're navigating are outside of both the shell and cylinder, so again, we can consider both of them as just one infinite line with a collective  linear charge listed in the last question.\\
Now, let's move the constants outside to make our integral look a bit nicer.

\[
	V= -2k\lambda \int^{a}_{b} \frac{1}{l} dl
\]
According to integration rules, we know the integral of $\frac{1}{l}dl$ is equal to $\ln(r)$\\
Let's substitute that into our equation and we get:\\
\[
	V=-2k\lambda\ln(l)\bigg|^{a}_{b}
\]

Now we have the equation for electric potential for a single point from a to b. In this case, we are evaluating the electric potential of a point, so we go from b=infinity to a= the distance away from the cylinder. \\
So evaluating V for both points we get:\\
\begin{align*}
	V_{P} &= -2k\lambda\ln(l)\bigg|^a_{b}\\
	V_{P} &= -2k\lambda\ln(l)\bigg|^{\sqrt{(0.46m)^2 + (0.46m)^2} }_{\infty}\\
	V_{P} &= -2 * 9 * 10^9 \frac{Nm^2}{C^2} * -0.275 \frac{\mu C}{m} * 10^-6 \frac{C}{\mu C} \ln(l)\bigg|^{0.65m}_{-\infty}\\
	V_{P} &= 4950 \frac{Nm}{C} \ln(l)\bigg|^{0.65m}_{\infty}\\
	V_{P} &= 4950 \frac{Nm}{C}\ln(0.65m) -0\\
	V_{P} &= 4950 \frac{Nm}{C}* -0.43m\\
	V_{P} &= -2128 \frac{N}{C}
\end{align*}

You probably noticed that when I pugged infinity into the equation on the fifth line, I set the value to 0. That's because it's generally given that infinity distance away, the electric potential is 0, it's kind of wonky. \\
Regardless, we now do the same math but for R.\\ 

\begin{align*}
	V_{R} &= -2k\lambda\ln(l)\bigg|^a_{b}\\
	V_{R} &= -2k\lambda\ln(l)\bigg|^{0.46m}_{\infty}\\
	V_{R}&= -2 * 9 * 10^9 \frac{Nm^2}{C^2} * -0.275 \frac{\mu C}{m} * 10^-6 \frac{C}{\mu C} \ln(l)\bigg|^{0.46m}_{\infty}\\
	V_{R} &= 4950 \frac{Nm}{C} \ln(l)\bigg|^{0.46m}_{\infty}\\
	V_{R} &= 4950 \frac{Nm}{C}\ln(0.46m) -0\\
	V_{R} &= 4950 \frac{Nm}{C}* -0.777m\\
	V_{R} &= -3846 \frac{N}{C}
\end{align*}

Now, to solve the problem, we subtract $V_{P}$ by $V_{R}$\\
\begin{align*}
	V_{dif} &= V_{P}-V_{R}\\
	V_{dif} &= -2128 \frac{N}{C} - (-3846 \frac{N}{C})\\
	V_{dif} &= 1718 \frac{N}{C}
\end{align*}

\section{Potential outer surfaces}
So, to solve for the potential on the outer surface of the conductor, it's pretty similar to what we did above, just instead of distance to the point, its the outer radius of the shell. 
\begin{align*}
	V_{Shell} &= -2k\lambda\ln(l)\bigg|^a_{b}\\
	V_{Shell} &= -2k\lambda\ln(l)\bigg|^{0.196m}_{\infty}\\
	V_{Shell}&= -2 * 9 * 10^9 \frac{Nm^2}{C^2} * -0.275 \frac{\mu C}{m} * 10^-6 \frac{C}{\mu C} \ln(l)\bigg|^{0.196m}_{\infty}\\
	V_{Shell} &= 4950 \frac{Nm}{C} \ln(l)\bigg|^{0.196m}_{\infty}\\
	V_{Shell} &= 4950 \frac{Nm}{C}\ln(0.196m) -0\\
	V_{Shell} &= 4950 \frac{Nm}{C}* -1.63m\\
	V_{Shell} &= -8068.5 \frac{N}{C}
\end{align*}

It gets a bit more complicated when looking at the insulator. So prepare for a slightly scuffed explanation.\\

When we look at the potential from an insulator, we have to visualize what would happen along the path as we go from infinity away to on the insulator.\\ By doing this, we would go through initially the electric field from the combined shell and cylinder until we reach the outer surface of the shell. Then, the electric field inside of the conducting shell from the outer to inner surface is 0. Then, we have an electric field as we go from the inner surface to the insulator. This is caused because the insulator has an electric field. So, in order to calculate our V, we have to sum everything.
\[
	V_{total} = V_{outside} + V_{inside Conductor} + V_{insulator} 
\]
Thankfully, $V_{outside}$ is solved for above and $V_inside Conductor$ has no electric field so it's zero. So all we need to do is solve for the insulator. We also know the $\lambda$ value already of the cylinder, so really it's just plug and chug of our infinite charge voltage equation.\\
Also, we need to change the bounds of our integral this time, since when we're calculating the potential for only the inductor, the distance would be from the inner surface of the inductor to the outer surface of the conductor.

\begin{align*}
	V_{insulator} &= -2k\lambda\ln(l)\bigg|^a_{b}\\
	V_{insulator} &= -2k\lambda\ln(l)\bigg|^{0.041m}_{0.166m}\\
	V_{insulator}&= -2 * 9 * 10^9 \frac{Nm^2}{C^2} * 0.195 \frac{\mu C}{m} * 10^-6 \frac{C}{\mu C} \ln(l)\bigg|^{0.041m}_{0.166m}\\
	V_{insulator} &= -3510 \frac{Nm}{C} \ln(l)\bigg|^{0.041m}_{0.116m}\\
	V_{insulator} &= -3510 \frac{Nm}{C}\ln(0.041m) -  (-3510 \frac{Nm}{C}\ln(0.166m)) \\
	V_{insulator} &= -3510 \frac{Nm}{C}* -3.19m - ( -3510 \frac{Nm}{C}* -1.79m)\\
	V_{insulator} &= 4914 \frac{N}{C}
\end{align*}

Now that we have the V of the insulator, we can pug it into our V total.
\begin{align*}
	V_{total} &= V_{outside} + V_{inside conductor} + V_{insulator}\\
	V_{total} &= -8068.5 \frac{N}{C} + 0 + 4914 \frac{N}{C}\\
	V_{total} &= -3154.5 \frac{N}{C}
\end{align*}

Now, our question is asking for the potential difference between the outer surface of the conductor to the insulator, so we subtract $V_{shell}$ by $V_{total} $
\begin{align*}
	V(e) - V(a) &= V_{shell} - V_{total}\\
	V(e) - V(a) &= -8068.5 \frac{N}{C} - (-3154.5 \frac{N}{C})\\
	V(e) - V(a) &= 4914 \frac{N}{C}
\end{align*}

\end{document}

